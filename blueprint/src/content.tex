% In this file you should put the actual content of the blueprint.
% It will be used both by the web and the print version.
% It should *not* include the \begin{document}
%
% If you want to split the blueprint content into several files then
% the current file can be a simple sequence of \input. Otherwise It
% can start with a \section or \chapter for instance.

\section{$H$-polytopes}

\begin{definition}
    \label{def:h_polytope}
    \uses{def:primspace, def:bounded}
    
\end{definition}

\begin{proposition}
    \label{prop:h_is_v}
    Every $H$-polytope is a $V$-polytope.
    \uses{def:h_polytope, def:v_polytope}
\end{proposition}

\subsection{Primitive Spaces}

\begin{definition}
    \label{def:primspace}
    \lean{Primspace} % reference structure, not def Set.is_primspace
    \leanok
\end{definition}

\begin{lemma}
    \label{lem:ambient_is_primspace}
    \lean{AffineSpace.ambient_is_primspace}
    \uses{def:primspace}
    \leanok
\end{lemma}

\begin{proof}
    
\end{proof}

\begin{lemma}
    \label{lem:empty_is_primspace}
    \lean{AffineSpace.empty_is_primspace}
    \uses{def:primspace}
    \leanok
\end{lemma}

\begin{proof}
    % if you add \uses{} here, and all those results are proved, 
    % it gets marked as "ready to be formalized"
\end{proof}

\begin{lemma}
    \label{lem:subspace_restriction_is_primspace}

    The intersection of a primspace in $A$ with an affine subspace $E$ of $A$ is itself a primspace in $E$ (but not necessarily in $A$).

    \lean{Primspace.subspace_restriction_is_primspace}
    \leanok
\end{lemma}

\begin{proof}
    \uses{def:primspace}
    \leanok
\end{proof}

\subsection{Bounded}

\begin{definition}
    \label{def:bounded}
    % maybe we can use ConvexBody.isBounded from the Analysis library
\end{definition}

\section{$V$-Polytopes}

\begin{definition}
    \label{def:v_polytope}
    \uses{def:convex_hull}
    A $V$-polytope is the convex hull of finitely many points.
\end{definition}

\begin{proposition}
    \label{prop:v_is_h}
    \uses{def:v_polytope, def:h_polytope, prop:h_is_v, prop:dual_dual_is_id}
\end{proposition}

\subsection{Convex Hulls}

\begin{definition}
    \label{def:is_convex}
    A set $S$ is convex if for any $x, y \in S$, and any $t \in [0,1]$, $S$ also contains $t x + (1-t) y$.
    \lean{Set.is_convex}
    \leanok
\end{definition}

\begin{lemma}
    \label{lem:empty_is_convex}
    \lean{AffineSpace.empty_is_convex}
    The empty set is convex.
    \leanok
\end{lemma}

\begin{proof}
    \uses{def:is_convex}
    \leanok
\end{proof}

\begin{lemma}
    \label{lem:ambient_is_convex}
    \lean{AffineSpace.ambient_is_convex}
    $\AA^n$ is convex.
    \leanok
    %% not sure this is exactly equivalent to the lean statement
    %% --> what should be here?
\end{lemma}

\begin{proof}
    \uses{def:is_convex}
    \leanok
\end{proof}

\begin{definition}
    \label{def:convex_hull}
    \lean{convex_hull}
    The convex hull of a set $A$ is the intersection of all convex subsets containing $A$.
    \leanok
    \uses{def:is_convex}
\end{definition}

\begin{proposition}
    \label{prop:convex_hull_is_convex}  
    \lean{convex_hull.is_convex}
    \uses{def:convex_hull, def:is_convex}
    \leanok
\end{proposition}

\begin{proof}
    \leanok
\end{proof}

\begin{lemma}
    \label{lem:ch_of_empty_is_empty}
    \lean{convex_hull.of_empty_is_empty}
    \uses{def:convex_hull}
    \leanok
\end{lemma}

\begin{proof}
    \uses{lem:empty_is_convex}
    \leanok
\end{proof}

\subsection{Polar duals}

\begin{definition}
    \label{def:polar_dual}
    \uses{def:v_polytope, def:h_polytope}
\end{definition}

\begin{proposition}
    \label{prop:dual_dual_is_id}
    \uses{def:polar_dual}
\end{proposition}

\section{The Main theorem}

\begin{theorem}
    \label{thm:h_v_equivalent}
    \uses{def:h_polytope, def:v_polytope, prop:h_is_v, prop:v_is_h}
    Every $H$-polytope is a $V$-polytope and vice-versa.
\end{theorem}